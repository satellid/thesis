\chapter{Appendix: Source Code}
\label{apx:source-code}

The actual or recent Satellid code repository with Meteor including Git, is structured like this directory tree:

\dirtree{%
.1 satellid-meteor.
.2 .git.
.2 .meteor.
.2 .snippets.
.2 both.
.3 app.coffee.
.3 collections.coffee.
.2 client.
.3 client.coffee.
.2 packages.
.2 private.
.3 knowledge.json.
.2 server.
.3 server.coffee.
.2 styles.
.3 app.stylus.
.2 templates.
.3 app.jade.
.3 templates.jade.
.2 tests.
.3 cucumber.
.4 ....
.2 start.sh.
}

\noindent The following listings are the main source code of Satellid in Meteor, both server side and client side.
Some inside these included codes are combined together and truncated.
Actual code are mainly using preprocessors.
Main logic code is using CoffeeScript for JavaScript (listing \autoref{lst:satellid-code-coffee}), main view code is using Jade for HTML (listing \autoref{lst:satellid-code-jade}), and main style code is using Stylus for CSS (listing \autoref{lst:satellid-code-styl}).
The complete and most updated source code is available publicly on GitHub at \url{https://github.com/satellid/satellid-meteor}.

\begin{listing}[tbp]
  \caption[Satellid main logic code]{Satellid main logic code CoffeeScript (*.coffee)}
  \inputminted{coffeescript}{\dir/code/.snippets/satellid.coffee}
  \label{lst:satellid-code-coffee}
\end{listing}

\begin{listing}[tbp]
  \caption[Satellid main view code]{Satellid main view code using Jade (*.jade)}
  \inputminted{jade}{\dir/code/.snippets/satellid.jade}
  \label{lst:satellid-code-jade}
\end{listing}

\begin{listing}[tbp]
  \caption[Satellid main style code]{Satellid main style code using Stylus (*.stylus)}
  \inputminted{css}{\dir/code/.snippets/satellid.styl}
  \label{lst:satellid-code-styl}
\end{listing}

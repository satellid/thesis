%\pdfbookmark[0]{Abstract}{abstract}

\begingroup
\let\clearpage\relax
\let\cleardoublepage\relax

\label{chap:abstract}
\chapter{Abstract}

\textbf{\myName.} \myNPM \\
\textbf{\myTitle} \\
%\myDepTitle \\
\textbf{\myThesisType}. \myDepartmentLong, \myFacultyLong, \myUni, \myYear. \\
Keywords: \myKeywords \\
(77 + xv + appendix)

\hfill

\singlespacing

% Background
Knowledge in human perspective mostly always defined as the thinking foundation within everyday life,
especially when people think and remember their daily personal knowledge.
Daily personal knowledge are things in context of personal profiles, companies, products, places, events, and others.
People use mostly their own brain for storing them, then retrieve and recall that in the memory when it needed.
Knowledge management is the de facto way to help manage them, because the brain mostly cannot handle tons of them all quickly and safely.
It can be as simple as writing the knowledge onto a paper or book, further as complex as storing the knowledge into a software or database.
Unfortunately, the problems of managing those are often remain still.
There are tons of them that must be managed, but they can’t be handled with regular tools.
Especially when the knowledge are unstructured or has some various contexts.

% Purpose
The primary purpose of this study is dedicated to discuss and explain the problem then provide and develop the solution in managing daily knowledge.
This writing develop a proposed knowledge manager named Satellid, for the easy and innovative way of managing them which make use of contextual system.
Users can identify and classify then organize those knowledge by using data and template approach.
Data behaves as the content of the knowledge whereas template behaves as the structure of the knowledge.
The essential interaction with the system is to browse, read, edit, add, and delete (BREAD) knowledge.
As the initial implementation, it made as a web application.
The technologies behind it are NoSQL document database called MongoDB with its JSON universal data or file format, web technologies especially JavaScript programming language, and Node.js platform with a full-stack framework called Meteor.
% Result
After the implementation is done, Satellid can be used for collecting and managing various daily personal knowledge, for the embetterment of knowledge management.

\onehalfspacing

\hfill

%\noindent \myTitle

\noindent Bibliography (2001-2015)

%\pagebreak
%{
%\selectlanguage{indonesian}
%\chapter*{Abstrak}
%}

\selectlanguage{american}

\endgroup

\vfill
